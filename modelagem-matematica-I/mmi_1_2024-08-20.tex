\documentclass{article}
\usepackage{graphicx} 
\usepackage{amsmath} 
\usepackage{tcolorbox}
\usepackage{mdframed} 
\usepackage{array} 
\usepackage{caption}
\usepackage{float}

\title{Modelagem Matemática I \\ 
        \large Primeiros exercícios de programação çlinear}
\author{Matheus Gabriel}
\date{Agosto 2024}

\begin{document}

\maketitle

\begin{tcolorbox}[colback=yellow!10!white, colframe=yellow!75!black, title=Aviso]
Não consegui anotar as explicações dos exercícios, então tenho apenas os resultados finais. 
\end{tcolorbox}

\section{Primeiro problema}

\subsection{Enunciado}
\begin{mdframed}[backgroundcolor=yellow!10, linecolor=black, linewidth=2pt]
\textbf{A Marcenaria (recriação do enunciado original)}

Uma marcenaria fabrica dois tipos de móveis: mesas e armários. Cada mesa gera um lucro de 4, enquanto cada armário gera 1 de lucro. Devido a restrições de produção, a marcenaria precisa maximizar o lucro total, respeitando as seguintes condições:
Para cada mesa produzida, são necessários 2 metros de madeira, e para cada armário, 3 metros de madeira. No total, a marcenaria possui até 12 metros de madeira disponíveis.
O tempo de trabalho necessário para produzir uma mesa é de 2 horas, e para produzir um armário é de 1 hora. A marcenaria possui um total de 8 horas de trabalho disponíveis.

Determine quantas mesas e quantos armários a marcenaria deve produzir para maximizar o lucro, respeitando as restrições de materiais e tempo.
\end{mdframed}

\subsection{Modelagem}

Considere o seguinte problema de programação linear:

\begin{align*}
    \text{Maximizar} \quad & Z = 4x_1 + x_2 \\
    \text{Sujeito às restrições:} \quad & \\
    & 2x_1 + 3x_2 \leq 12 \\
    & 2x_1 + x_2 \leq 8 \\
    & x_1 \geq 0 \\
    & x_2 \geq 0
\end{align*}

Onde:
\begin{itemize}
    \item \(x_1\) representa o número de mesas.
    \item \(x_2\) representa o número de armários.
\end{itemize}

\subsection{Resposta}

\begin{tcolorbox}[colback=blue!5!white, colframe=blue!75!black, title=Resultado]
A solução ótima é obtida quando:
\begin{align*}
    x_1^* &= 4 \\
    x_2^* &= 0
\end{align*}
E o valor máximo da função objetivo \(Z\) é:
\begin{align*}
    Z^* &= 4x_1^* + x_2^* \\
    Z^* &= 4 \cdot 4 + 0 \\
    Z^* &= 16
\end{align*}
\end{tcolorbox}

\section{Segundo problema}

\subsection{Enunciado}

\begin{mdframed}[backgroundcolor=yellow!10, linecolor=black, linewidth=2pt]
\textbf{Fábrica de sapatos e botinas}

Uma fábrica produz sapatos e botinas. Através da tabela abaixo, formule um modelo que maximize o lucro da fábrica.
\end{mdframed}

\begin{center}
\begin{tabular}{|c|c|c|c|}
\hline
\textbf{Matéria Prima} & \textbf{Sapatos} & \textbf{Botinas} & \textbf{Disponibilidade} \\
\hline
Couro & 2 & 1 & 8 \\
\hline
Borracha & 1 & 2 & 7 \\
\hline
Cola & 0 & 1 & 3 \\
\hline
\textbf{Lucro por unidade (em R\$)} & \textbf{1} & \textbf{1} & \textbf{-} \\
\hline
\end{tabular}
\end{center}

\subsection{Modelagem}

Modelando o problema temos:
\\
Onde:
\begin{itemize}
    \item \(x_1\) representa os valores do couro.
    \item \(x_2\) representa os valores da borracha.
\end{itemize}

\begin{tcolorbox}[colback=yellow!10!white, colframe=red!75!black, title=Importante]
Apesar de existirem três itens na coluna de matéria prima, a cola não conta como \(x_3\)
\end{tcolorbox}

\begin{align*}
    \text{Maximizar} \quad & L = 1x_1 + 1x_2 \\
    \text{Sujeito às restrições:} \quad & \\
    & 2x_1 + x_2 \leq 8 \\
    & x_1 + 2x_2 \leq 7 \\
    & x_2 \leq 3 \\
    & x_1 \geq 0 \\
    & x_2 \geq 0
\end{align*}

\subsection{Resposta}


\begin{tcolorbox}[colback=blue!5!white, colframe=blue!75!black, title=Resultado]
A solução ótima é obtida quando:
\begin{align*}
    x_1^* &= 3 \\
    x_2^* &= 2
\end{align*}
E o valor máximo da função objetivo \(L\) é:
\begin{align*}
    L^* &= 5
\end{align*}
\end{tcolorbox}


\clearpage

\section{Terceiro problema}
\subsection{Enunciado}
\begin{mdframed}[backgroundcolor=yellow!10, linecolor=black, linewidth=2pt]
    \textbf{Lucro de vendas}
    
    Um fabricante de bombons tem estocado bombons de chocolate, sendo 130kg com recheio de cerejas e 170kg com recheio de menta. Ele decide vender o estoque na forma de dois pacotes sortidos diferentes. Um pacote contém uma mistura com metade do peso em bombons de cereja e metade em menta e vende por R\$ 20,00 por kg. O outro pacote contpem uma mistura de um terço de bombons de cereja e dois terços de menta e vende por R\$ 12,50 por kg. O vendedor deveria preparar quantos quilos de cada mistura a fim de maximizar o seu lucro de vendas?
\end{mdframed}

\subsection{Modelagem}
Modelando o problema temos:

\begin{align*}
    \text{Maximizar} \quad & L = 20x_1 + 12,5x_2 \\
    \text{Sujeito às restrições:} \quad & \\
    & \frac{x_1}{2} + \frac{x_2}{3} \leq 130 \\
    & \frac{x_1}{2} + \frac{2x_2}{3}\leq 170 \\
    & x_1 \geq 0 \\
    & x_2 \geq 0
\end{align*}


\begin{itemize}
    \item \(x_1\) representa o Kg do pacote 1.
    \item \(x_2\) representa o Kg do pacote 2.
\end{itemize}


\subsection{Resposta}

\begin{tcolorbox}[colback=blue!5!white, colframe=blue!75!black, title=Resultado]
A solução ótima é obtida quando:
\begin{align*}
    x_1^* &= 260 \\
    x_2^* &= 0
\end{align*}
E o valor máximo da função objetivo \(L\) é:
\begin{align*}
    L^* &= 5200
\end{align*}
\end{tcolorbox}

\end{document}
