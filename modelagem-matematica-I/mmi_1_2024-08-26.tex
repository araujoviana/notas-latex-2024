\documentclass{article}
\usepackage{graphicx} % Requerido para inserção de imagens
\usepackage{amsmath} % Requerido para fórmulas matemáticas
\usepackage{geometry} % Para ajustar as margens
\geometry{a4paper, margin=1in} % Margens padrão para A4
\usepackage{csquotes}
\usepackage{tcolorbox}
\usepackage{svg}

\title{Modelagem Matemática I}
\author{Matheus Gabriel}
\date{Agosto 2024}

\begin{document}

\maketitle

\section{Minimizando o Custo}

\subsection{Enunciado}
Um estudante quer projetar um desjejum com flocos de milho e leite que seja o  mais econômico possível. Levando em conta o que ele consegue comer nas  suas outras refeições, ele decide que seu café da manhã deveria supri-lo com 9  gramas de proteínas, pelo menos uma terça parte da necessidade diária  recomendada (NDR) de vitamina D e pelo menos uma quarta parte da NDR de  cálcio. Ele encontra as seguintes informações nutricionais nas embalagens do  leite e dos flocos de milho:

\begin{figure}[h]
    \centering
    \includegraphics[width=0.5\linewidth]{image.png}
\end{figure}

A fim de não ter uma mistura muito empapada ou muito seca, o estudante  decide limitar-se a misturas que contenham no mínimo 1 e no máximo 3 xícaras  de flocos de milho por copo de leite. Quais quantidades de leite e de flocos de  milho ele deve utilizar para minimizar o custo de seu desjejum?
% Inserir o enunciado do problema aqui.

\subsection{Modelagem e Resolução}

Definimos as variáveis de decisão como:
\begin{itemize}
    \item $x_1$: Quantidade de leite em copos (meia quantidade).
    \item $x_2$: Quantidade de flocos de milho em xícaras.
\end{itemize}

Como dito no enunciado: \textquote{Quais quantidades de leite e de flocos de  milho ele deve utilizar para \textbf{minimizar o custo} de seu desjejum?}, então a função objetivo a ser minimizada é:
\begin{equation}
    z = 7.5x_1 + 50x_2
\end{equation}

Sujeito às seguintes restrições estabelecidas pelo enunciado:


\begin{align*}
    &\text{Pelo menos 9g de proteína} 
    &\quad \Longrightarrow \quad&
    4x_1 + 2x_2 \geq 9 \\
    &\text{Pelo menos $\frac{1}{3}$NDR de vitamina D} 
    &\quad \Longrightarrow \quad&
    \frac{1}{8}x_1 + \frac{1}{10}x_2 \geq \frac{1}{3} \text{ NDR} \\
    &\text{Pelo menos $\frac{1}{4}$NDR de Cálcio} 
    &\quad \Longrightarrow \quad&
    \frac{1}{6}x_1 \geq \frac{1}{4} \text{ NDR} \\
    &\text{Pelo menos 1 xícara de flocos de milho por copo (dois meios copos) de leite} 
    &\quad \Longrightarrow \quad&
    \frac{x_2}{x_1} \geq \frac{1}{2} \\
    &\text{No máximo 3 xícaras de flocos de milho por copo (dois meios copos) de leite} 
    &\quad \Longrightarrow \quad&
    \frac{x_2}{x_1} \leq \frac{3}{2} \\
    &\text{Como se trata de um cenário real assumimos valores positivos} 
    &\quad \Longrightarrow \quad&
    x_1 \geq 0, \; x_2 \geq 0
\end{align*}

Além disso, a razão entre a quantidade de flocos de milho e a quantidade de leite deve satisfazer:
\begin{equation}
    1 \leq \frac{x_2}{2x_1} \leq 3
\end{equation}


\subsection{Encontrando a região viável}

\begin{tcolorbox}[colback=yellow!10, colframe=red!50!black, title=Importante!]
\textquote{A região viável é sempre a interseção de um número finito de retas e planos.}
\end{tcolorbox}

\subsubsection{Primeira Inequação}

Consideramos a inequação:
\begin{align}
    \frac{1}{2}x_1 + \frac{1}{3}x_2 &\leq 130 \\
    \frac{1}{2}x_1 + \frac{1}{3}x_2 &= 130
\end{align}
Se $x_1 = 0$, então $x_2 = 390$. \\
Se $x_2 = 0$, então $x_1 = 260$.

\subsubsection{Segunda Inequação}

Consideramos a inequação:
\begin{align}
    \frac{1}{2}x_1 + \frac{2}{3}x_2 &\leq 170 \\
    \frac{1}{2}x_1 + \frac{2}{3}x_2 &= 170
\end{align}
Se $x_1 = 0$, então $x_2 = 255$. \\
Se $x_2 = 0$, então $x_1 = 340$.

\begin{figure}[h]
    \centering
    \includesvg[width=\textwidth]{desmos-graph.svg}
    \caption{Gráfico gerado a partir das duas inequações, nota-se que a região viável é onde as duas cores se sobrepoem no quadrante superior direito.}
\end{figure}

\subsection{Encontrando o Ponto de Interseção das Inequações}

Consideramos as duas equações:
\begin{align}
    \frac{1}{2}x_1 + \frac{1}{3}x_2 &= 130 \quad \text{(I)} \\
    \frac{1}{2}x_1 + \frac{2}{3}x_2 &= 170 \quad \text{(II)}
\end{align}

Subtraindo (I) de (II):
\begin{align}
    \left(\frac{1}{2}x_1 + \frac{2}{3}x_2\right) - \left(\frac{1}{2}x_1 + \frac{1}{3}x_2\right) &= 170 - 130 \\
    \frac{2}{3}x_2 - \frac{1}{3}x_2 &= 40 \\
    \frac{1}{3}x_2 &= 40 \\
    \mathbf{x_2 = 120}
\end{align}

Substituindo $x_2 = 120$ em (I):
\begin{align}
    \frac{1}{2}x_1 + \frac{1}{3} \cdot 120 &= 130 \\
    \frac{1}{2}x_1 + 40 &= 130 \\
    \frac{1}{2}x_1 &= 90 \\
    \mathbf{x_1 = 180}
\end{align}

Então o ponto de interseção se encontra em $(180x_1,120x_2)$.
\end{document}
