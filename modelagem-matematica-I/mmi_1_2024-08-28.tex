\documentclass{article}
\usepackage{graphicx} % Requerido para inserção de imagens
\usepackage{amsmath} % Requerido para fórmulas matemáticas
\usepackage{geometry} % Para ajustar as margens
\geometry{a4paper, margin=1in} % Margens padrão para A4
\usepackage{csquotes}
\usepackage{tcolorbox}
\usepackage{svg}

\title{Modelagem Matemática I\\
    \large Exemplo com resolução gráfica}
\author{Matheus Gabriel}
\date{Agosto 2024}

\begin{document}

\maketitle

\section{Exemplo}

\subsection{Modelagem e Resolução}

\begin{equation}
    \mathbf{\text{Maximizar } z = 7.5x_1 + 50x_2}
\end{equation}

Sujeito às seguintes restrições:


\begin{align*}
\begin{cases}
    2x_1 + x_2 \leq 8 \\
    x_1 + 2x_2 \leq 7 \\
    x_2 \leq 3 \\
    x_1 \geq 0 \text{ e } x_2 \geq 0
\end{cases}
\end{align*}


\subsection{Encontrando pra $x_1 = 0$ ou $x_2 = 0$}

\begin{tcolorbox}[colback=yellow!10, colframe=red!50!black, title=Importante!]
\textquote{A região viável é sempre a interseção de um número finito de retas e planos.}
\end{tcolorbox}

\subsubsection{Primeira Inequação}

Consideramos a inequação:
\begin{equation}
    2x_1 + x_2 = 8 \\    
\end{equation}
\begin{table}[h]
    \centering
    \begin{tabular}{c|c}
        $x_1$ & $x_2$ \\
        $0$ & $8$ \\
        $4$ & $0$
    \end{tabular}
\end{table}
Se $x_1 = 0$, então $x_2 = 8$. \\
Se $x_2 = 0$, então $x_1 = 4$.

\subsubsection{Segunda Inequação}

Consideramos a inequação:
\begin{equation}
    x_1 + 2x_2 = 7 \\    
\end{equation}
\begin{table}[h]
    \centering
    \begin{tabular}{c|c}
        $x_1$ & $x_2$ \\
        $0$ & $3,5$ \\
        $7$ & $0$
    \end{tabular}
\end{table}
Se $x_1 = 0$, então $x_2 = 3,5$. \\
Se $x_2 = 0$, então $x_1 = 7$.

\subsection{Terceira inequação}
\begin{equation}
    x_2 \leq 3
\end{equation}
Essa é a mais simples, se trata de uma linha reta horizontal no 3.
\subsection{O gráfico}
É essencial destacar e nomear as vértices do polígono.
\begin{figure}[h]
    \centering
    \includegraphics[width=1\linewidth]{desmos-graph.png}
\end{figure}

\subsection{Encontrando o ponto ideal}
\begin{tcolorbox}[colback=yellow!10, colframe=red!50!black, title=Importante!]
O ponto que satisfaz a maximização é uma das vértices da região viável!
\end{tcolorbox}

\subsubsection{Definindo $x_1$ e $x_2$}
\begin{align}
    2x_1 + x_2 = 8 \ (-2) \\
    x_1 + 2x_2 = 7 \\
    \rule{2cm}{0,4pt}  \\
    -3x_1 = -9 \\
    x_1 = 3 \\
    x_2 = 2
\end{align}

\subsubsection{Encontrando o ponto mais viável}
\begin{align}
    A = (0,0) : Z = 0 \\
    B = (0,3) : Z = 3 \\
    C = (1,3) : Z = 4 \\
    \mathbf{D = (3,2) : Z = 5} \\
    E = (4,0) : Z = 4
\end{align}

Então o ponto mais viável é o $D = (3,2)$.
\end{document}
