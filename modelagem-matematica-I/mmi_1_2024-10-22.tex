\documentclass{article}
\usepackage{graphicx} % Requerido para inserção de imagens
\usepackage{amsmath} % Requerido para fórmulas matemáticas
\usepackage{geometry} % Para ajustar as margens
\geometry{a4paper, margin=1in} % Margens padrão para A4
\usepackage{csquotes}
\usepackage{tcolorbox}
\usepackage{svg}
\usepackage{xcolor}


\title{Modelagem Matemática I\\
    \large Algoritmo do Transporte}
\author{Matheus Gabriel}
\date{Outubro 2024}

\begin{document}

\maketitle

\section{Métodos para gerar a solução inicial}
\begin{enumerate}
    \item Método do Custo Mínimo
    \item Método do Canto Noroeste
\end{enumerate}
\section{Método Noroeste gerado pelo GPT}

\documentclass{article}
\usepackage{amsmath}
\usepackage{amsfonts}
\usepackage{graphicx}

\begin{document}

\title{Método do Noroeste para Problemas de Transporte}
\author{}
\date{}
\maketitle

\subsection{Introdução}
O Método do Noroeste é uma técnica utilizada para encontrar uma solução inicial para problemas de transporte. O objetivo é minimizar os custos de transporte, distribuindo a carga de forma eficiente.

\subsection{Passo a Passo}

\subsubsection{1. Montar a tabela de custos}
\begin{itemize}
    \item Liste os fornecedores e os consumidores.
    \item Crie uma tabela com os custos de transporte entre cada fornecedor e consumidor.
    \item Insira as ofertas (quantidades disponíveis) dos fornecedores e as demandas (quantidades necessárias) dos consumidores.
\end{itemize}

\subsubsection{2. Selecionar o canto noroeste}
\begin{itemize}
    \item Comece pelo canto noroeste da tabela (célula na posição [1,1]).
\end{itemize}

\subsubsection{3. Atribuir o máximo possível}
\begin{itemize}
    \item Atribua o máximo possível de unidades para a célula selecionada, que é o mínimo entre a oferta do fornecedor e a demanda do consumidor.
    \item Atualize a oferta e a demanda, subtraindo a quantidade atribuída.
\end{itemize}

\subsubsection{4. Verificar se a oferta ou a demanda foi satisfeita}
\begin{itemize}
    \item Se a oferta do fornecedor for satisfeita, mova-se para a próxima linha.
    \item Se a demanda do consumidor for satisfeita, mova-se para a próxima coluna.
\end{itemize}

\subsubsection{5. Repetir o processo}
\begin{itemize}
    \item Continue a atribuir unidades nas células adjacentes, sempre movendo-se para o canto noroeste da célula que acabou de ser preenchida, até que todas as ofertas e demandas sejam atendidas.
\end{itemize}

\subsubsection{6. Finalizar a tabela}
\begin{itemize}
    \item Ao finalizar, verifique se todas as ofertas e demandas foram atendidas.
    \item A tabela final mostrará a quantidade de unidades a serem transportadas entre cada fornecedor e consumidor.
\end{itemize}

\subsection{Conclusão}
O Método do Noroeste é um método simples e eficiente para encontrar uma solução inicial em problemas de transporte. A partir dessa solução, é possível aplicar métodos de otimização para minimizar ainda mais os custos.

\end{document}

