\documentclass{article}
\usepackage{graphicx} % Requerido para inserção de imagens
\usepackage{amsmath} % Requerido para fórmulas matemáticas
\usepackage{geometry} % Para ajustar as margens
\geometry{a4paper, margin=1in} % Margens padrão para A4
\usepackage{csquotes}
\usepackage{tcolorbox}
\usepackage{svg}
\usepackage{xcolor}


\title{Modelagem Matemática I\\
    \large Método Simplex em prática}
\author{Matheus Gabriel}
\date{Agosto 2024}

\begin{document}

\maketitle

\section{Exercício 1}

\subsection{Enunciado}
Maximize: $Z =2x_1 + x_2$

Sujeito às restrições:
\begin{align*}
    3x_1+4x_2 \leq 6 \\
    6x_1+x_2 \leq 3 \\
    x_1,x_2 \geq 0
\end{align*}

\subsection{Modelagem e resolução}
O objetivo é encontrar a solução ótima, através de várias soluções viáveis. Para encontrar a linha pivô existe uma linha de decisão,  mas não pensei em anotar.

\textbf{Encontramos a solução ótima quando não existem mais valores negativos na linha Z da tabela.}
\begin{align*}
Z - 2x_1-x_2 = 0 \\
\\
    3x_1 + 4x_2 + x_3 + 0x_4 = 6 \\
    6x_1 + x_2 + 0x_3 + x_4 = 3 \\
    x_1,x_2,x_3,x_4 \geq 0
\end{align*}

Solução inicial básica = $(0,0,6,3)$

Agora colocando isso tudo na tabela:

\begin{table}[h!]
    \centering
    \begin{tabular}{ c | c c c c | c c}
       base  & $x_1$ & $x_2$ & $x_3$ & $x_4$ & $b$ \\
       \hline
        $x_2$ & 3 & 4 & 1 & 0 & 6 \\
        $x_4$ & 6 & 1 & 0 & 1 & 3 \\
        \hline
        $Z$ & -2 & -1 & 0 & 0 & 0 \\
    \end{tabular}
\end{table}
As linhas aqui são denominadas como L1, L2 e L3.

"a" e "b" são coordenadas na tabela, o "a" representa a parte dos xis:
\begin{align*}
    \boxed{\frac{b_1}{a_11} = \frac{6}{3} = 2} \\
    \boxed{\frac{b_2}{a_21} = \frac{3}{6} = \frac{1}{2}} \\
\end{align*}

Em um problema real, você escreveria várias tabelas de acordo com o seu progresso.

Agora fazendo isso:
\begin{align*}
    L_1 \leftarrow L_1 - 3L_2 \\
    L_3 \leftarrow L_3 + 2L_2
\end{align*}

Aplicamos para a tabela
\begin{table}[h!]
    \centering
    \begin{tabular}{ c | c c c c | c c}
       base  & $x_1$ & $x_2$ & $x_3$ & $x_4$ & $b$ \\
       \hline
        $x_3$ & 3 & 4 & 1 & 0 & 6 \\
         $x_4$ & 1 & $\frac{1}{6}$ & 0 & $\frac{1}{6}$ & $\frac{1}{2}$ \\
         \hline
        $Z$ & -2 & -1 & 0 & 0 & 0 \\
    \end{tabular}
\end{table}

\clearpage

E então,
\begin{table}[h!]
    \centering
    \begin{tabular}{ c | c c c c | c c}
       base  & $x_1$ & $x_2$ & $x_3$ & $x_4$ & $b$ \\
       \hline
        $x_3$ & 0 & $\frac{7}{2}$ & 1 & $-\frac{1}{2}$ & $\frac{9}{2}$ \\
         $x_1$ & 1 & $\frac{1}{6}$ & 0 & $\frac{1}{6}$ & $\frac{1}{2}$ \\
        \hline
        $Z$ & 0 & $-\frac{2}{3}$ & 0 & $\frac{1}{3}$ & 1 \\
    \end{tabular}
\end{table}

Analisando a tabela, vemos quais valores x têm valores usáveis para a solução, como no caso do $x_1$ e $x_3$, que valem 1, então colocamos eles na solução viável.
Nova solução viável: $= (\frac{1}{2}, 0, \frac{9}{2}, 0)$

Vamos formar uma nova tabela:
\begin{align*}
    \boxed{\frac{b_1}{a_21} = \frac{9}{2} \cdot \frac{2}{7} = \frac{9}{7}} \\
    \frac{b_2}{a_22} = \frac{1}{2} \cdot \frac{6}{1} = 3
\end{align*}

\begin{table}[h!]
    \centering
    \begin{tabular}{ c | c c c c | c c}
       base  & $x_1$ & $x_2$ & $x_3$ & $x_4$ & $b$ \\
       \hline
        $x_3$ & 0 & $1$ & $\frac{2}{7}$ & $-\frac{1}{7}$ & $\frac{9}{7}$ \\
         $x_1$ & 1 & $\frac{1}{6}$ & 0 & $\frac{1}{6}$ & $\frac{1}{2}$ \\
        \hline
        $Z$ & 0 & $-\frac{2}{3}$ & 0 & $\frac{1}{3}$ & 1 \\
    \end{tabular}
\end{table}
\begin{align*}
    L_2 \leftarrow L_2 - \frac{1}{6}L_1 \\
    -\frac{1}{6} \cdot \frac{2}{7} = - \frac{1}{21} \\
    \frac{1}{5} + \frac{1}{6} \cdot \frac{1}{7} = \frac{1}{6}[1+\frac{1}{7}] = \frac{1}{6} \cdot \frac{8}{7} = \frac{4}{21} \\
    \frac{1}{2} - \frac{1}{6} \cdot \frac{9}{7} = \frac{1}{2} - \frac{3}{14} = \frac{4}{14} = \frac{2}{7} \\
    L_3 \leftarrow L_3 + \frac{2}{3}L_1 \\
    \frac{2}{3} \cdot \frac{2}{7} = \frac{4}{21} \\
    \frac{1}{3} - \frac{2}{3} \cdot \frac{1}{7} = \frac{1}{3}[1-\frac{2}{7}] = \frac{1}{3} \cdot \frac{5}{7} = \frac{5}{21} \\
    1+\frac{2}{3} \cdot \frac{9}{7} = 1 + \frac{6}{7} = \frac{13}{7}
\end{align*}
\begin{table}[h!]
    \centering
    \begin{tabular}{ c | c c c c | c c}
       base  & $x_1$ & $x_2$ & $x_3$ & $x_4$ & $b$ \\
       \hline
        $x_2$ & 0 & $1$ & $\frac{2}{7}$ & $-\frac{1}{7}$ & $\frac{9}{7}$ \\
         $x_1$ & 1 & 0 & $-\frac{1}{21}$ & $\frac{4}{21}$ & $\frac{2}{7}$ \\
        \hline
        $Z$ & 0 & $0$ & $\frac{4}{21}$ & $\frac{5}{21}$ & $\frac{13}{7}$ \\
    \end{tabular}
\end{table}

Solução ótima $\mathbf{= (\frac{2}{7}, \frac{9}{7}})$
\begin{align*}
    \boxed{Z = \frac{13}{7}}
\end{align*}
\end{document}
