\documentclass{article}
\usepackage{graphicx} % Required for inserting images
\usepackage{tcolorbox}
\usepackage{hyperref}
\usepackage{listings}
\usepackage{xcolor}
\usepackage{minted}
\usepackage{fancyvrb}

\title{Organização de Computadores \\
    \large Microcontroladores MIPS}
\author{Matheus Gabriel}
\date{Agosto 2024}

\begin{document}

\maketitle

\section{Introdução}

\begin{tcolorbox}[colback=blue!10, colframe=blue!80, title=Seção menos importante, fonttitle=\bfseries]
    Pode pular essa seção...
\end{tcolorbox}

\subsection{Definição de microcontrolador}

É um circuito integrado que reúne em seu interior um
microprocessador, memórias (voláteis e não voláteis),
temporizadores, conversor A/D, diversas portas para E/S,
requerendo apenas uma fonte de alimentação externa para
funcionar.

\subsection{Aplicações}

\begin{itemize}
    \item Automação industrial
    \item Indústria automotiva
    \item Eletrodomésticos
    \item Produtos eletrônicos
    \item Impressoras, mouses e teclados
    \item Etc...
\end{itemize}

\clearpage

\section{MIPS}
\subsection{Especificidades}
\subsubsection{Uso}
Essa arquitetura \textbf{RISC} está presente em sistemas como:
\begin{itemize}
    \item Video Games
    \item Roteadores da Cisco
    \item Impressoras
    \item Etc...
\end{itemize}

\subsubsection{Instruções}
As instruções possuem tamanho fixo \textbf{32 ou 64 bits}, e são mais fáceis de manipular, especialmente pelo \textit{pipeline}.

\begin{tcolorbox}[colback=orange!10, colframe=orange!80, title=Interessante, fonttitle=\bfseries]
    Na arquitetura x86 as instruções não tem tamanho fixo.
\end{tcolorbox}

As instruções seguem o seguinte padrão:

\begin{minted}[linenos]{asm}
# OPERAÇÃO <lista de operandos>
add $s1, $s2, $s3    
\end{minted}

\subsubsection{Registradores}
Registradores são pequenas memórias temporárias localizadas dentro da 
própria CPU, capazes de armazenar alguns bits.

A quantidade de bits depende da arquitetura, no caso do MIPS os registradores têm \textbf{32 bits}. 

MIPS possui 32 registradores de uso geral, numerados de 0 a 31, sendo geralmente referenciados pelo seu nome. Exemplo: \mintinline{asm}{$t0, $s2}, etc.

Seguir as convenções do uso de registradores seguida por programadores e
compiladores ajuda na compatibilidade entre outros microcontroladores MIPS.

\clearpage

\subsection{Convenção dos Registradores do MIPS}


\begin{table}[h!]
\centering
\begin{tabular}{|c|c|l|}
\hline
\textbf{Nome} & \textbf{Número} & \textbf{Uso} \\ \hline
\mintinline{asm}{$zero} & 0 & Valor constante 0 \\ \hline
\mintinline{asm}{$at}   & 1 & Registrador temporário reservado para o assembler \\ \hline
\mintinline{asm}{$v0}   & 2 & Registrador para resultados de chamadas de sistema e funções \\ \hline
\mintinline{asm}{$v1}   & 3 & Registrador para resultados de chamadas de sistema e funções \\ \hline
\mintinline{asm}{$a0}   & 4 & Argumento 0 \\ \hline
\mintinline{asm}{$a1}   & 5 & Argumento 1 \\ \hline
\mintinline{asm}{$a2}   & 6 & Argumento 2 \\ \hline
\mintinline{asm}{$a3}   & 7 & Argumento 3 \\ \hline
\mintinline{asm}{$t0}   & 8 & Registrador temporário (não preservado) \\ \hline
\mintinline{asm}{$t1}   & 9 & Registrador temporário (não preservado) \\ \hline
\mintinline{asm}{$t2}   & 10 & Registrador temporário (não preservado) \\ \hline
\mintinline{asm}{$t3}   & 11 & Registrador temporário (não preservado) \\ \hline
\mintinline{asm}{$t4}   & 12 & Registrador temporário (não preservado) \\ \hline
\mintinline{asm}{$t5}   & 13 & Registrador temporário (não preservado) \\ \hline
\mintinline{asm}{$t6}   & 14 & Registrador temporário (não preservado) \\ \hline
\mintinline{asm}{$t7}   & 15 & Registrador temporário (não preservado) \\ \hline
\mintinline{asm}{$s0}   & 16 & Registrador salvo (preservado) \\ \hline
\mintinline{asm}{$s1}   & 17 & Registrador salvo (preservado) \\ \hline
\mintinline{asm}{$s2}   & 18 & Registrador salvo (preservado) \\ \hline
\mintinline{asm}{$s3}   & 19 & Registrador salvo (preservado) \\ \hline
\mintinline{asm}{$s4}   & 20 & Registrador salvo (preservado) \\ \hline
\mintinline{asm}{$s5}   & 21 & Registrador salvo (preservado) \\ \hline
\mintinline{asm}{$s6}   & 22 & Registrador salvo (preservado) \\ \hline
\mintinline{asm}{$s7}   & 23 & Registrador salvo (preservado) \\ \hline
\mintinline{asm}{$t8}   & 24 & Registrador temporário (não preservado) \\ \hline
\mintinline{asm}{$t9}   & 25 & Registrador temporário (não preservado) \\ \hline
\mintinline{asm}{$k0}   & 26 & Registrador reservado para o kernel \\ \hline
\mintinline{asm}{$k1}   & 27 & Registrador reservado para o kernel \\ \hline
\mintinline{asm}{$gp}   & 28 & Registrador do ponteiro global \\ \hline
\mintinline{asm}{$sp}   & 29 & Registrador do ponteiro da pilha \\ \hline
\mintinline{asm}{$fp}   & 30 & Registrador do ponteiro de quadro \\ \hline
\mintinline{asm}{$ra}   & 31 & Registrador de retorno da função \\ \hline
\end{tabular}
\end{table}

\clearpage

\subsection{Instruções Aritméticas}

Cada instrução no MIPS executa uma única operação. Sempre que escrevemos instruções, manipulamos registradores para armazenar dados. 

As instruções no MIPS possuem dois operandos e um destino para guardar o resultado. 

Por exemplo: \mintinline{asm}{operação $resultado, $operando1, $operando2}.

\begin{table}[h!]
\centering
\begin{tabular}{|l|l|l|}
\hline
\textbf{Sintaxe} & \textbf{Nome} & \textbf{O que faz} \\ \hline
\mintinline{asm}{add $d, $s, $t} & ADD & Soma \$s e \$t, resultado em \$d. \\ \hline
\mintinline{asm}{sub $d, $s, $t} & SUB & Subtrai \$t de \$s, resultado em \$d. \\ \hline
\mintinline{asm}{mul $d, $s, $t} & MUL & Multiplica \$s e \$t, resultado em \$d. \\ \hline
\mintinline{asm}{div $s, $t} & DIV & Divide \$s por \$t. Quociente em \$lo, resto em \$hi. \\ \hline
\mintinline{asm}{addi $t, $s, imm} & ADDI & Soma \$s e imediato, resultado em \$t. \\ \hline
\mintinline{asm}{subi $t, $s, imm} & SUBI & Subtrai imediato de \$s, resultado em \$t. \\ \hline
\mintinline{asm}{muli $t, $s, imm} & MULI & Multiplica \$s por imediato, resultado em \$t. \\ \hline
\mintinline{asm}{divi $s, $t, imm} & DIVI & Divide \$s por imediato. Quociente em \$lo, resto em \$hi. \\ \hline
\end{tabular}
\end{table}

\end{document}
