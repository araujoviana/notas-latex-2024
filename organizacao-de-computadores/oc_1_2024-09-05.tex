\documentclass{article}
\usepackage{graphicx} % Required for inserting images
\usepackage{tcolorbox}
\usepackage{hyperref}
\usepackage{listings}
\usepackage{xcolor}
\usepackage{minted}
\usepackage{fancyvrb}

\title{Organização de Computadores \\
    \large Conjunto de Instruções MIPS}
\author{Matheus Gabriel}
\date{Agosto 2024}

\begin{document}

\maketitle

\section{Tipos de dados}

\begin{itemize}
    \item \textbf{byte}: Define um dado como 1 byte (8 bits)
    \item \textbf{double}: Número de ponto flutuante de 64 bits
    \item \textbf{float}: Número de ponto flutuante de 32 bits
    \item \textbf{half}: Define um dado de 2 bytes (16 bits)
    \item \textbf{word}: int de 4 bytes
    \item \textbf{space x}: Reserva x bytes em memória (Data Segment)
\end{itemize}

\section{Tipos de instruções no MIPS}
No MIPS existem três categorias de instruções: 

\begin{itemize}
    \item \textbf{TIPO R}: (Formato para \textbf{R}egistradores): Maioria das instruções aritméticas e lógicas.
    \item \textbf{TIPO I}: (Aritmética \textbf{I}mediata): Instruções de transferência de dados, desvios condicionais (branch) e operações imediatas.
    \item \textbf{TIPO J}: (Operações de Jump/Deslocamento): Instruções de desvios incondicionais.
\end{itemize}


\section{Convertendo um código Assembly para código de máquina}

\begin{table}[]
    \centering
    \begin{tabular}{|c |c| c| c| c| c|}
        op & rs & rd & rt & shamt & funct \\
        6 bits & 5 bits & 5 bits & 5 bits & 5 bits & 6 bits
    \end{tabular}
\end{table}

\subsection{Exemplo}

Convertemos transformando os bits em binário:
\begin{enumerate}
    
    \item add \$t0, \$s0, \$s1 (add rd,rs,rt)
    \item 000000 10000 10001 01000 00000 100000
    \item 0x02114020
\end{enumerate}
A representação binária é dividida em grupos de 4, e convertida para hexadecimal.
\end{document}
