\documentclass{article}
\usepackage{graphicx} % Required for inserting images
\usepackage{tcolorbox}
\usepackage{hyperref}
\usepackage{listings}
\usepackage{xcolor}
\usepackage{minted}
\usepackage{fancyvrb}

\title{Organização de Computadores \\
    \large Alocação Dinâmica de Memória}
\author{Matheus Gabriel}
\date{Outubro 2024}

\begin{document}

\maketitle

\section{Memória (Heap)}

A memória é separada em:
\begin{enumerate}
    \item Stack
    \item Heap
    \item Data (variáveis não incializadas)
    \item Data (variáveis inicializadas)
    \item Text (Instruções do Programa Executável)
\end{enumerate}

O Heap é a usada para alocação dinâmica, que é feita durante a execução, isso inclui estruturas de dados dinâmicas, porém, é necessário liberar a memória após o uso para evitar vazamentos de memória.

\end{document}
