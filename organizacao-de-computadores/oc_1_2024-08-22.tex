\documentclass{article}
\usepackage{graphicx} % Required for inserting images
\usepackage{tcolorbox}
\usepackage{hyperref}
\usepackage{listings}
\usepackage{xcolor}
\usepackage{minted}

\title{Organização de Computadores \\
    \large Programação em Assembly MIPS}
\author{Matheus Gabriel}
\date{Agosto 2024}

\begin{document}

\maketitle

\section{Introdução}

A linguagem Assembly é uma das formas mais diretas de interagir com a arquitetura de um computador, oferecendo um nível de controle próximo ao da linguagem de máquina. Neste semestre, focaremos na programação em Assembly para a arquitetura MIPS. Utilizaremos o IDE MARS (MIPS Assembler and Runtime Simulator) para a edição e execução dos nossos códigos.

\section{Programação básica}

Os passos aqui descritos não são necessariamente os mesmo passos de todo código Assembly (obviamente).
\subsection{O arquivo}
Para os exemplos discutidos neste documento, utilizamos o arquivo denominado \texttt{ExemploOperacoesAritmeticas.asm}, disponibilizado pelo professor na plataforma Moodle. Os comentários são do professor.
\begin{tcolorbox}[colback=yellow!10!white, colframe=yellow!75!black, title=Aviso]
O arquivo .asm do exemplo, e a maioria (senão todos) dos outros exemplos de código não são incluídos no repositório git, porém, em alguns documentos as explicações juntas já formam o código completo.
\end{tcolorbox}

\subsection{Definição de variáveis}

\begin{minted}[
linenos
]
{asm}
.data
# .data EH O SEGMENTO ONDE DECLARAMOS AS "VARIAVEIS"

N1: .asciiz "Digite o 1o numero: "
N2: .asciiz "Digite o 2o numero: "
MsgSoma: .asciiz "\nSoma = "
MsgSub: .asciiz "\nSubtracao = "
MsgMul: .asciiz "\nMultiplicacao = "
\end{minted}


\begin{itemize}
    \item \texttt{.data} indica o começo da seção onde são declaradas as variáveis
    \item \texttt{.asciiz} armazena a string no segmento de dados e adiciona um terminador nulo \texttt{\textbackslash0}.
\end{itemize}
Esse código seria equivalente à \texttt{MsgSoma = "Soma = "} na linguagem Python

\subsection{Atribuição de valores aos registradores e entrada do usuário}
\begin{minted}[
linenos
]{asm}
.text
# .text EH O SEGMENTO QUE CONTEM AS INSTRUCOES/PROGRAMA

# EM ASSEMBLY, A SYSCALL EH UTILIZADA PARA VARIAS ACOES: LER DADOS DO TECLADO, 
# IMPRIMIR ALGO NA TELA, FINALIZAR O PROGRAMA, ETC

# IMPRIMIR A MENSAGEM N1 NA TELA
li $v0,4 # ATRIBUINDO O VALOR 4 PARA O REG $v0 
# (COD 4: INDICA PRINT DE STRING P/ SYSCALL)
la $a0,N1 # CARREGANDO O END. DA VAR N1 PARA DENTRO DO REGISTRADOR a0
# syscall #SYSCALL CHAMA O SO PARA EXECUTAR A ACAO RELATIVA AO CODIGO QUE ESTA NO $v0

# LER UM INTEIRO DIGITADO PELO USUARIO

li $v0,5 # ATRIBUINDO O VALOR 5 PARA O REG $v0 
# (COD 5: INDICA INPUT DE INT P/ SYSCALL) 
syscall

# FAZER UMA COPIA DO DADO LIDO
move $t0, $v0 # COPIANDO O CONTEUDO DE $v0 PARA $t0


# IMPRIMIR A MENSAGEM N2 NA TELA
li $v0,4 # ATRIBUINDO O CONTEÚDO DO $v0 PARA 4 (PRINT)
la $a0,N2
syscall

# LER UM INTEIRO DIGITADO PELO USUARIO
li $v0,5 # ATRIBUINDO O VALOR 5 PARA O REG $v0 
# (COD 5: INDICA INPUT DE INT P/ SYSCALL) 
syscall

# FAZER COPIA DO DADO LIDO
move $t1, $v0 # COPIANDO O CONTEUDO DE $v0 PARA $t1
\end{minted}

\begin{itemize}
    \item \texttt{.text} indica o começo da seção que contém as instruções do programa
    \item \texttt{li} atribui uma \textbf{constante} (e.g 4, 2) para o registrador especificado.
    \item \texttt{la} atribui uma \textbf{variável} (e.g N1, MsgSoma) para o registrador especificado.
    \item \texttt{syscall} chama o sistema operacional para executar as ações definidas anteriormente.
    \item \texttt{move} copia o conteúdo de um registrador para outro.
\end{itemize}

Resumindo, esse trecho de código atribui os valores definidos em \texttt{.data} para os registradores para serem impressos depois. As constantes usadas em \texttt{li} são específicas para o tipo de syscall que o usuário deseja fazer, use a \href{https://courses.missouristate.edu/kenvollmar/mars/help/syscallhelp.html}{tabela de syscalls do MIPS} como referência.

\subsection{Cálculo das operações}

\begin{minted}[
linenos
]{asm}
# SOMA $t0 COM $t1 E ARMAZENA O RESULTADO EM $t2
add $t2, $t0, $t1
# add EH A OPERACAO DE ADICAO
# $t2 É O REGISTRADOR DE DESTINO
# $t0 E $t1 SÃO OS REGISTRADORES DAS VARIAVEIS N1 E N2

# IMPRIMIR A MENSAGEM DE SOMA = NA TELA
li $v0,4
la $a0,MsgSoma
syscall

# IMPRIMIR O RESULTADO NA TELA
li $v0,1
move $a0, $t2
syscall

# SUBTRACAO $t0 COM $t1 E ARMAZENA O RESULTADO EM $t2
sub $t2, $t0, $t1
# sub EH A OPERACAO DE SUBTRACAO
# $t2 É O REGISTRADOR DE DESTINO
# $t0 E $t1 SÃO OS REGISTRADORES DAS VARIAVEIS N1 E N2

# IMPRIMIR A MENSAGEM DE RESPOSTA NA TELA
li $v0,4
la $a0,MsgSub
syscall

# IMPRIMIR O RESULTADO NA TELA
li $v0,1
move $a0, $t2
syscall

# MULTIPLICACAO $t0 COM $t1 E ARMAZENA O RESULTADO EM $t2
mul $t2, $t0, $t1
# mul EH A OPERACAO DE MULTIPLICACAO
# $t2 É O REGISTRADOR DE DESTINO
# $t0 E $t1 SÃO OS REGISTRADORES DAS VARIAVEIS N1 E N2

# IMPRIMIR A MENSAGEM DE RESPOSTA NA TELA
li $v0,4
la $a0,MsgMul
syscall

# IMPRIMIR O RESULTADO NA TELA
li $v0,1
move $a0, $t2
syscall

li $v0, 10	# 10 É O COD. QUE A SYSCALL USA PARA ENCERRAR O PROGRAMA
syscall		# SYSCALL CHAMA O SO PARA EXECUTAR A ACAO RELATIVA AO CODIGO QUE ESTA NO $v0
\end{minted}

\begin{itemize}
    \item \texttt{add} realiza a operação de adição
    \item \texttt{sub} realiza a operação de subtração
    \item \texttt{mult} realiza a operação de multiplicação
\end{itemize}
Após realizar as operações especificadas, os resultados são atribuídos à registradores e seus valores são impressos na tela.
\end{document}
