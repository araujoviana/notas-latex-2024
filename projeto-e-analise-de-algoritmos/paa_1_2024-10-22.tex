\documentclass{article}
\usepackage{graphicx}
\usepackage{minted} 
\usepackage{mathtools}
\usepackage{amsmath}
\usepackage{amssymb}
\usepackage{hyperref}
\usepackage{cancel}
\usepackage{xcolor}
\usepackage{tcolorbox}
\newcommand{\lp}{\Bigl(}
\newcommand{\rp}{\Bigr)}
\title{Projeto e Análise de Algoritmos \\
\large Continuação de Heap Sort}
\author{Matheus Gabriel}
\date{Outubro de 2024}

\begin{document}

\maketitle

\section{Heap de Máximo (Heap-max)}

Eles possuem a propriedade de que \textbf{}

\subsection{Pseudocódigo}

O \texttt{i} se refere ao índice (começando por 1) do vetor.

\begin{verbatim}
    Parent(i)
        return [i/2]

    Left(i)
        return 2i

    Right(i)
        return 2i+1
\end{verbatim}

\section{Afunda (Max-Heapify)}

Essa operação é o coração do Heap Sort.

Suponha um "quase heap-max" cujo único problema seja a raiz, \textit{ou seja}, a raiz não preenche a propriedade max-heap: $A[\text{Parent}(i)] \geq A[i]$

O \textit{afunda} trabalha para recolocar a raiz (correta) em sua posição final.


\begin{tcolorbox}[colback=yellow!10!white, colframe=orange!80!black, title=Aviso]
    É recomendável visualizar o algoritmo de Heap Sort usando um visualizador interativo para entender melhor seu funcionamento e a estrutura de dados envolvida.
\end{tcolorbox}

\textbf{Ordenar o Heap Max != Criar o Heap Max}
\end{document}
