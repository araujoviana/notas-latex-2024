\documentclass{article}
\usepackage{graphicx}
\usepackage{minted} 
\usepackage{mathtools}
\usepackage{amsmath}
\usepackage{amssymb}
\usepackage{hyperref}
\usepackage{cancel}
\usepackage{xcolor}
\usepackage{tcolorbox}
\newcommand{\lp}{\Bigl(}
\newcommand{\rp}{\Bigr)}
\title{Projeto e Análise de Algoritmos \\
\large Heap Sort}
\author{Matheus Gabriel}
\date{Outubro de 2024}

\begin{document}

\maketitle

\section{Árvores}

\subsection{Definição}

Árvores são grafos acíclicos conexos, ou seja:
\begin{enumerate}
    \item Não possuem ciclos de conexão
    \item Não possuem partes soltas, são compostas por uma árvore inteira única (múltiplas árvores são uma \textit{floresta})
\end{enumerate}

\subsection{Composição}

A árvore é composta por:

\begin{enumerate}
    \item Raiz
    \item Nós internos
    \item Folhas
\end{enumerate}

\section{Heap Sort}

\subsection{Definição}

Um algoritmo que tira proveito de uma estrutura de dados para obter melhoria assintótica em relação aos algoritmos de ordenação quadráticos no tempo.

Esse é o \textbf{Heap de máximo}, com isso, o Heap Sort consegue rodar em tempo $O(n \lg n)$ e \textbf{localmente}.

\subsection{Etapas da implementação}

Certas etapas são importantes para manter o tempo de execução em $n \lg n$, geralmente usaríamos uma estrutura de dados auto referencial, porém, não será necessário criar ela separadamente, tudo é \textbf{local}.

\subsubsection{Heaps}

A estrutura de dados Heap (binário) se trata de um \textbf{vetor} que pode ser visto como uma árvore binária quase completa.

Cada nó dela corresponde a um elemento do vetor.

A árvore está preenchida em todos os níveis exceto, possivelmente, no mais baixo, que é preenchido a partir da esquerda.

Importante e vale pesquisar: \texttt{Heap\_Max}

\end{document}
