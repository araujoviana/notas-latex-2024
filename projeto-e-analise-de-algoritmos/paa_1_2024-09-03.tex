\documentclass{article}
\usepackage{graphicx}
\usepackage{minted} 
\usepackage{mathtools}
\usepackage{amsmath}
\usepackage{amssymb}
\usepackage{hyperref}
\usepackage{cancel}
\usepackage{xcolor}
\usepackage{tcolorbox}

\title{Projeto e Análise de Algoritmos \\
\large Continuação de Algoritmos de Ordenação}
\author{Matheus Gabriel}
\date{Agosto de 2024}

\begin{document}

\maketitle

\section{Continuação com exercícios de Análise Assintótica}
\subsection{Exercício 1}
\subsubsection{Enunciado}
Coloque em ordem assintoticamente crescente:

\begin{itemize}
    \item $300$
    \item $n^2$
    \item $10^90$
    \item $O(1)$
    \item $n$
    \item $n^3$
    \item $n \lg n$
    \item $\lg n$
    \item $n^n$
    \item $\frac{n}{2}$
    \item $n!$
    \item $n^2 \lg n$
\end{itemize}
\subsubsection{Resposta}
\begin{align*}
    300 = 10^90 = O(1) < \lg n = 20 \lg n = O(\lg n) \\
    < n = \frac{n}{2} = O(n) < O(n \lg n) < O(n^2) \\
    O(n^2 \lg n) < O(n^3) < O(2^n) < O(n!) < O(n^n)
\end{align*}

\subsection{Exercício 2}
\subsubsection{Enunciado}
Encontre a fórmula fechada para:
\begin{align*}
    T(1) = 1 \\
    T(n) = 2T(n-1) + 1
\end{align*}
\subsubsection{Resposta}
Mas:
\begin{align*}
    T(n-1) = 2T(n-2)+1 \\
    \therefore T(n) = 2(2T(n-2)+1) +1 \\
    T(n) = 2^2 T(n-2)+2+1
\end{align*}
Mas:
\begin{align*}
    T(n-2) = 2T(n-2)+1 \\
    \therefore T(n) = 2^2(2T(n-3)+1) + 2 + 1 = 2^3 T(n-3) + 2 ^2 + 2^1+2^0 \\
    \text{Generalizando: } T(n) = 2^i T(n-i) + \sum^{(i^{-1})}_{k=0}2^k \\
    \text{E vai parar quando } n -i = 1 \rightarrow \boxed{i = n-1} \\
    \therefore T(n) = 2^{n-1} \cdot [T(1) \rightarrow 1] + \sum^{n-2}_{k=0}2^k=2^{n-1} + 2^{n-1} -1 = 2^n -1 \\
    \therefore \boxed{T(n) = \Theta (2^n)}
\end{align*}
\subsection{Exercício 3}
\subsubsection{Enunciado}
Idem para:
\begin{align*}
    T(1) = 1 \\
    T(n) = 4T(\frac{n}{2}) + n^2
\end{align*}
\subsubsection{Resposta}
Mas
\begin{align*}
    T(\frac{n}{2}) = 4T(\frac{n}{2^2})+(\frac{n}{2})^2 \\
    \therefore T(n) = 4(4T(\frac{n}{2^2} + \frac{n^2}{4}) + n^ = 4^2 T(\frac{n}{2^2}) + n^2 + n^2 \\
    \text{Mas } T(\frac{n}{2})  = 4T(\frac{n}{2^3}) + (\frac{n}{2^2})^2 \\
    \therefore T(n) = 4^2(4T(\frac{n}{2^3}) + \frac{n^2}{4^2}) + n^2 + n^2 = 4^3T(\frac{n}{2^3}) + n^2 + n^2 + n^2 \\
    \text{Generalizando, temos: } T(n) = 4^i T(\frac{n}{2^i}) + \sum^i_{k=1}n^2 \\
    \text{E vai parar quando } \frac{n}{2^i} = 1 \rightarrow \boxed{i = \log_2 n} \\
    \therefore T(n) = 4^{\log_2 n} \cdot [T(1) \rightarrow 1] + \sum^{\log_2 n}_{k=1}n^2 \\
    T(n) = n^{\log_2 4} + n^2 \log_2 n = n^2 + n^2 \lg n \\
    \boxed{T(n) = \Theta (n^2 \lg n)}
\end{align*}

Lembrando que $ a^{\log_b^c} = c^{\log_b^a}$
\end{document}