% Created 2024-11-01 sex 10:57
% Intended LaTeX compiler: pdflatex
\documentclass[11pt]{article}
\usepackage[utf8]{inputenc}
\usepackage[T1]{fontenc}
\usepackage{graphicx}
\usepackage{longtable}
\usepackage{wrapfig}
\usepackage{rotating}
\usepackage[normalem]{ulem}
\usepackage{amsmath}
\usepackage{amssymb}
\usepackage{capt-of}
\usepackage{hyperref}
\author{Matheus Gabriel}
\date{Novembro 2024}
\title{Projeto e Análise de Algorítmos I\\\medskip
\large Ordenando em O(n)}
\hypersetup{
 pdfauthor={Matheus Gabriel},
 pdftitle={Projeto e Análise de Algorítmos I},
 pdfkeywords={},
 pdfsubject={},
 pdfcreator={Emacs 29.4 (Org mode 9.7.11)}, 
 pdflang={English}}
\begin{document}

\maketitle
\tableofcontents

\section{Ordenando em O(n)}
\label{sec:orgbb0ca90}
Lembrando que o teorema original dizia:
\begin{quote}
Qualquer algoritmo de ordenação por comparação requer \(\omega(n \lg n)\) comparações no pior caso.
\end{quote}
\subsection{Problema de ordenação}
\label{sec:org4d234b2}

\subsubsection{Enunciado}
\label{sec:org8816c93}
Escreva um algoritmo de ordenação para ordenar números inteiros de 1 até 99999, não repetidos.
Este algoritmo deve rodar em \(O(n)\) para o pior caso.
\subsubsection{Minha hipótese}
\label{sec:org4ce77d2}
Simplesmente imprima os números de 1 a 99999
\subsubsection{Uma das soluções}
\label{sec:org3a0dc12}
\begin{enumerate}
\item Algoritmo para Ordenar Lista de 1 a 99,999 em O(n)
\label{sec:org5c1f8ce}

Este método ordena uma lista contendo números únicos de 1 a 99,999 em tempo O(n) usando uma lista auxiliar indexada diretamente.
\begin{enumerate}
\item Passos para Implementação
\label{sec:orgcba410b}
\begin{enumerate}
\item Inicialize uma lista auxiliar de tamanho 99,999. Cada posição representa um número na faixa de 1 a 99,999.
\item Percorra a lista original e insira cada número na posição correspondente na lista auxiliar:
\begin{itemize}
\item Exemplo: Coloque o número 1 na posição 0, o número 2 na posição 1, e assim por diante.
\end{itemize}
\item A lista auxiliar estará automaticamente ordenada após a inserção de todos os elementos.
\end{enumerate}
\item Implementação pequena em Python
\label{sec:org4739190}
\begin{verbatim}
import random

def counting_sort(arr):
    # Inicializa o contador com zeros para cada número entre 0 e 9
    count = [0] * 10
    output = [0] * len(arr)

    # Conta a ocorrência de cada elemento na lista
    for num in arr:
        count[num] += 1

    # Atualiza o contador para armazenar as posições dos elementos
    for i in range(1, 10):
        count[i] += count[i - 1]

    # Constrói a lista de saída de forma ordenada
    for num in reversed(arr):
        output[count[num] - 1] = num
        count[num] -= 1

    return output

# Gera uma lista aleatória de 0 a 9 sem repetições
arr = random.sample(range(10), 10)

# Aplica o Counting Sort
sorted_arr = counting_sort(arr)

return arr, sorted_arr
\end{verbatim}
\begin{enumerate}
\item Complexidade
\label{sec:org36df2ee}
\begin{itemize}
\item \textbf{Tempo}: O(n), pois percorremos a lista uma vez e inserimos cada elemento diretamente na posição.
\item \textbf{Espaço}: O(n), pois utilizamos uma lista auxiliar de tamanho 99,999.
\end{itemize}
\end{enumerate}
\end{enumerate}
\end{enumerate}
\end{document}
