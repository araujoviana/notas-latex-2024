\documentclass{article}
\usepackage{graphicx}
\usepackage{minted} 
\usepackage{mathtools}
\usepackage{amsmath}
\usepackage{amssymb}
\usepackage{hyperref}
\usepackage{cancel}
\usepackage{xcolor}
\usepackage{tcolorbox}

\title{Projeto e Análise de Algoritmos \\
\large Algoritmos de Ordenação}
\author{Matheus Gabriel}
\date{Agosto de 2024}

\begin{document}

\maketitle

\section{Merge Sort}
\subsection{Explicação}
O Merge Sort é um algoritmo de ordenação baseado na técnica de \textbf{divisão e conquista}. O algoritmo segue os seguintes passos:

\begin{enumerate}
    \item \textbf{Divisão:} Divide o array (ou lista) em duas metades.
    \item \textbf{Conquista:} Ordena cada metade recursivamente.
    \item \textbf{Combinação:} Mescla as duas metades ordenadas para produzir o array ordenado final.
\end{enumerate}

A eficiência do Merge Sort é garantida pela sua abordagem recursiva e pela técnica de mesclagem, que resulta numa complexidade de tempo de \(\Theta(n \log n)\) tanto no pior caso quanto no melhor caso. Especificamente, a complexidade no pior caso é \(\Omega(n \log n)\), o que significa que o Merge Sort é garantidamente eficiente na ordenação de qualquer conjunto de dados.

\subsection{Exemplo visual}

\begin{minipage}{\textwidth}
    \includegraphics[width=0.5\linewidth]{image.png}
\end{minipage}
Passos do Merge Sort. Fonte: \href{https://www.programiz.com/dsa/merge-sort}{Programiz}.


\subsection{Código em C}
\subsubsection{Implementação}
\begin{minted}{c}
#include <stdio.h>
#include <stdlib.h>

void merge(int arr[], int l, int m, int r) {
    int n1 = m - l + 1;
    int n2 = r - m;

    int* L = (int*)malloc(n1 * sizeof(int));
    int* R = (int*)malloc(n2 * sizeof(int));

    for (int i = 0; i < n1; i++)
        L[i] = arr[l + i];
    for (int j = 0; j < n2; j++)
        R[j] = arr[m + 1 + j];

    int i = 0, j = 0, k = l;
    while (i < n1 && j < n2) {
        if (L[i] <= R[j]) {
            arr[k++] = L[i++];
        } else {
            arr[k++] = R[j++];
        }
    }

    while (i < n1) {
        arr[k++] = L[i++];
    }

    while (j < n2) {
        arr[k++] = R[j++];
    }

    free(L);
    free(R);
}

void mergeSort(int arr[], int l, int r) {
    if (l < r) {
        int m = l + (r - l) / 2;
        mergeSort(arr, l, m);
        mergeSort(arr, m + 1, r);
        merge(arr, l, m, r);
    }
}

void printArray(int arr[], int size) {
    for (int i = 0; i < size; i++)
        printf("%d ", arr[i]);
    printf("\n");
}

int main() {
    int arr[] = {12, 11, 13, 5, 6, 7};
    int arr_size = sizeof(arr) / sizeof(arr[0]);

    printf("Array original:\n");
    printArray(arr, arr_size);

    mergeSort(arr, 0, arr_size - 1);

    printf("\nArray ordenado:\n");
    printArray(arr, arr_size);
    return 0;
}
\end{minted}

\subsection{Código em Common Lisp}
\subsubsection{Implementação}
\begin{minted}{lisp}
(defun merge (left right)
  (cond
    ((null left) right)
    ((null right) left)
    ((<= (car left) (car right))
     (cons (car left) (merge (cdr left) right)))
    (t (cons (car right) (merge left (cdr right))))))

(defun merge-sort (list)
  (if (or (null list) (null (cdr list)))
      list
      (let* ((mid (/ (length list) 2))
             (left (subseq list 0 mid))
             (right (subseq list mid)))
        (merge (merge-sort left) (merge-sort right)))))

;; Exemplo de uso
(let ((unsorted-list '(12 11 13 5 6 7)))
  (format t "Lista original: ~a~%" unsorted-list)
  (format t "Lista ordenada: ~a~%" (merge-sort unsorted-list)))
\end{minted}

\section{Encontrando a fórmula fechada}
\begin{tcolorbox}[
  colback=yellow!10!white, 
  colframe=red!75!black,  
  fonttitle=\bfseries,     
  coltitle=black,         
  sharp corners=south,     
  boxrule=1mm,             
  title=Seção Importante,  
  width=\textwidth         
]
  Essa é uma das partes mais importantes da matéria!
\end{tcolorbox}
\subsection{Primeiro Exemplo (Mais explicativo)}

Vamos resolver a recorrência dada por:

\begin{align*}
    T(n) &= 2T\left(\frac{n}{2}\right) + \Theta(n) \\
    T(n) &= 2T\left(\frac{n}{2}\right) + n \\
    T(1) &= 1
\end{align*}

\textbf{Solução da Recorrência}

Para resolver a recorrência \(T(n) = 2T\left(\frac{n}{2}\right) + n\), utilizaremos o método de expansão:

1. \textbf{Expandindo a Recorrência:}

   Começamos substituindo \(T\left(\frac{n}{2}\right)\) na expressão de \(T(n)\):

   \begin{align*}
       T\left(\frac{n}{2}\right) &= 2T\left(\frac{n}{4}\right) + \frac{n}{2} \\
       T(n) &= 2 \left[2T\left(\frac{n}{4}\right) + \frac{n}{2}\right] + n \\
       T(n) &= 2^2T\left(\frac{n}{4}\right) + 2 \cdot \frac{n}{2} + n \\
       T(n) &= 2^2T\left(\frac{n}{4}\right) + n + n
   \end{align*}

   Continuamos expandindo para o próximo nível:

   \begin{align*}
       T\left(\frac{n}{4}\right) &= 2T\left(\frac{n}{8}\right) + \frac{n}{4} \\
       T(n) &= 2^2 \left[2T\left(\frac{n}{8}\right) + \frac{n}{4}\right] + n + n \\
       T(n) &= 2^3T\left(\frac{n}{8}\right) + 2^2 \cdot \frac{n}{4} + n + n \\
       T(n) &= 2^3T\left(\frac{n}{8}\right) + n + n + n
   \end{align*}

2. \textbf{Generalizando:}

   Observamos que após \(i\) expansões, temos:

   \begin{align*}
       T(n) &= 2^i T\left(\frac{n}{2^i}\right) + \sum_{k=0}^{i-1} 2^k \cdot \frac{n}{2^k}
   \end{align*}

   O somatório pode ser simplificado para:

   \begin{align*}
       \sum_{k=0}^{i-1} n &= n \cdot i
   \end{align*}

   Então:

   \begin{align*}
       T(n) &= 2^i T\left(\frac{n}{2^i}\right) + n \cdot i
   \end{align*}

3. \textbf{Determinação do Valor de \(i\):}

   A recursão termina quando \( \frac{n}{2^i} = 1 \), ou seja:

   \begin{align*}
       \frac{n}{2^i} &= 1 \\
       2^i &= n \\
       i &= \log_2 n
   \end{align*}

4. \textbf{Substituindo \(i\) na Fórmula Geral:}

   Quando \( i = \log_2 n \), temos:

   \begin{align*}
       T(n) &= 2^{\log_2 n} T(1) + n \cdot \log_2 n \\
       T(n) &= n \cdot 1 + n \cdot \log_2 n \\
       T(n) &= n + n \log_2 n
   \end{align*}

5. \textbf{Complexidade Assintótica:}

   Combinando os termos, obtemos:

   \begin{align*}
       T(n) &= \Theta(n \log_2 n)
   \end{align*}

Portanto, a fórmula fechada para a recorrência é:

\begin{align*}
    \boxed{T(n) = \Theta(n \log n)}
\end{align*}

\subsection{Segundo Exemplo}

Vamos resolver a seguinte recorrência:

\begin{align*}
    T(n) &= T\left(\frac{n}{2}\right) + 1 \\
    \text{com } T(1) &= 1
\end{align*}

\textbf{Solução da Recorrência}

Para resolver a recorrência \(T(n) = T\left(\frac{n}{2}\right) + 1\), utilizaremos o método de expansão:

1. \textbf{Expandindo a Recorrência:}

   Começamos substituindo \(T\left(\frac{n}{2}\right)\) na expressão de \(T(n)\):

   \begin{align*}
       T\left(\frac{n}{2}\right) &= T\left(\frac{n}{2^2}\right) + 1 \\
       T(n) &= T\left(\frac{n}{2^2}\right) + 1 + 1 \\
       T(n) &= T\left(\frac{n}{2^2}\right) + 2
   \end{align*}

   Continuamos expandindo para o próximo nível:

   \begin{align*}
       T\left(\frac{n}{2^2}\right) &= T\left(\frac{n}{2^3}\right) + 1 \\
       T(n) &= T\left(\frac{n}{2^3}\right) + 1 + 1 + 1 \\
       T(n) &= T\left(\frac{n}{2^3}\right) + 3
   \end{align*}

2. \textbf{Generalizando:}

   Observamos que após \(i\) expansões, temos:

   \begin{align*}
       T(n) &= T\left(\frac{n}{2^i}\right) + i
   \end{align*}

   A recursão termina quando \( \frac{n}{2^i} = 1 \), ou seja:

   \begin{align*}
       \frac{n}{2^i} &= 1 \\
       2^i &= n \\
       i &= \log_2 n
   \end{align*}

3. \textbf{Substituindo \(i\) na Fórmula Geral:}

   Quando \( i = \log_2 n \), temos:

   \begin{align*}
       T(n) &= T(1) + \log_2 n \\
       T(n) &= 1 + \log_2 n
   \end{align*}

4. \textbf{Complexidade Assintótica:}

   O termo constante \(1\) pode ser desconsiderado na notação assintótica, então obtemos:

   \begin{align*}
       T(n) &= \Theta(\log n)
   \end{align*}

Portanto, a fórmula fechada para a recorrência é:

\begin{align*}
    \boxed{T(n) = \Theta(\log n)}
\end{align*}

\end{document}