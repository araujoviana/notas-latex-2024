\documentclass{article}
\usepackage{graphicx}
\usepackage{minted} 
\usepackage{mathtools}
\usepackage{amsmath}
\usepackage{amssymb}
\usepackage{hyperref}
\usepackage{cancel}
\usepackage{xcolor}
\usepackage{tcolorbox}
\newcommand{\lp}{\Bigl(}
\newcommand{\rp}{\Bigr)}
\title{Projeto e Análise de Algoritmos \\
\large Teorema Mestre E Mais Problemas Com Recursão}
\author{Matheus Gabriel}
\date{Agosto de 2024}

\begin{document}

\maketitle

\section{Problema com recorrência}
\subsection{Enunciado}
Resolver: 
\begin{align*}
\begin{cases}
T(n) = 8T(\frac{n}{2}) + n \\
T(1) = 1
\end{cases}
\end{align*}

\subsection{Resolução}
\begin{align*}
    \text{Mas }T(\frac{n}{2}) = 8T(\frac{n}{2^2}) + \frac{n}{2} \\
    \therefore T(n) =  8(8T(\frac{n}{2^2})+\frac{n}{2}) + n \\
    T(n) = 8^2 T(\frac{n}{2^2}) + 4n + n \\
    \text{Mas } T(\frac{n}{2^2}) = 8T(\frac{n}{2^3}) + \frac{n}{2^2} \\
    \therefore T(n) = 8^2(8T(\frac{n}{2^3}) + \frac{n}{2^2}) + 4n + n \\
    T(n) = 8^3T(\frac{n}{2^3}) + 16n + 4n + n \\
    T(n) = 8^3(\frac{n}{2^3}) + n (4^2+4^1+4^0) \\
\end{align*}
Generalizando, temos:
\begin{align*}
    T(n) = 8^i T(\frac{n}{2^i}) + n\sum^{i-1}_{k=0}4^k
\end{align*}
E vai parar quando 
\begin{align*}
    \frac{n}{2^i} = 1 \\
    \text{ou}\\
    \boxed{i = \log_2 n}
\end{align*}
Tendo em mente que $\boxed{\sum^n_{k=0}c^k=\frac{c^{n+1}-1}{c-1}}$, vamos finalizar:
\begin{align*}
    T(n) = 8^{\log_2 n}\cdot [T(1) \rightarrow 1] + n \sum^{\log_2 n - 1}_{k=0}4^k \\
    \text{Usando a fórmula dada:} \\
    T(n) = n^{\log_2 8} + n(\frac{4^{\log_2 n} - 1}{4-1}) \\
    T(n) = n^{3} + n(\frac{n^{\log_2 4} - 1}{3}) \\
    = n^3 + n(\frac{n^2 -1}{3}) \\
    T(n) = \frac{3n^3 + n^3 - n}{3} \\
    = \frac{4n^3 -n}{3} \\
    \therefore \boxed{T(n) = \Theta(n^3)}
\end{align*}

\section{Teorema mestre}
\subsection{O que é}
O método resolve recorrências da forma:
\begin{align*}
    T(n) = aT(\frac{n}{b}) + f(n)
\end{align*}
Ou seja,
\begin{itemize}
    \item \textbf{Use apenas em recorrências com divisão e conquista! }
    \item \textbf{É extremamente importante dizer o caso do teorema mestre usado!}
    \item \textbf{$\lg$ é sempre base 2}
\end{itemize}


Casos:
\begin{enumerate}
    \item Se $f(n) = O(n^{log_b a-\epsilon})$ para alguma constante $\epsilon > 0$, então $T(n) = \Theta(n^{\log_b a })$
    \item Se $f(n) = O(n^{log_b a})$, então $T(n) = \Theta(n^{\log_b a} \lg n)$.
    \item Se $f(n) = \Omega(n^{log_b a+\epsilon})$, para alguma constante $\epsilon > 0$, e se $af(\frac{n}{b}) \leq cf(n)$ para alguma constante $c > 1$ e $n$ suficientemente grande, então $T(n) = \Theta(f(n))$.
\end{enumerate}

\subsection{Exemplos}
\subsubsection{Primeiro exemplo}

\textit{Exemplo incompleto, consulte o segundo.}

Resolva para: $T(n) = 9T(\frac{n}{3}) + n $

Temos:
\begin{align*}
    a = 9\\
    b = 3 \\
    f(n) = n
\end{align*}
Então resolvemos como:
\begin{align*}
    n^{\log_b a} = n^{\log_3 9} = \Theta(n^2)
\end{align*}
Como $f(n) \dots$

\subsubsection{Segundo exemplo}
\begin{align*}
    T(n) = T(\frac{2n}{3}) + 1 \\
    a = 1 \\
    b = \frac{3}{2} \\
    f(n) = 1
\end{align*}
Vamos encontrar o caso:
\begin{align*}
    n^{\log_{\frac{3}{2}}1}= n^0 = 1
\end{align*}

Pelo \textit{caso 2} do teorema mestre:
\begin{align*}
    T(n) = \Theta(\lg n)
\end{align*}


\subsubsection{Terceiro exemplo}
\begin{align*}
    T(n) = 3T(\frac{n}{4}) + n \lg n \\
    a = 3 \\
    b = 4 \\
    f(n) = n \lg n
\end{align*}
Vamos encontrar o caso:
\begin{align*}
    n^{\log_4 3}= n^{0,8...}
\end{align*}

Pelo \textit{caso 3} do teorema mestre:
\begin{align*}
    T(n) = \Theta(n \lg n)
\end{align*}




\end{document}