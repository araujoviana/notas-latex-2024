\documentclass{article}
\usepackage{graphicx}
\usepackage{minted} 
\usepackage{mathtools}
\usepackage{amsmath}
\usepackage{amssymb}

\title{Projeto e Análise de Algoritmos}
\author{Matheus Gabriel}
\date{Agosto de 2024}

\begin{document}

\maketitle

\section{Verificando a notação O}

\subsection{Exemplo 1}

$2n + 10$ é $O(n)$? 

Podemos realizar uma manipulação para encontrar $c$ e $n_0$:

\begin{align}
    2n + 10 &\leq c \cdot n \\
    c \cdot n - 2n &\geq 10 \\
    (c - 2)n &\geq 10 \\
    n &\geq \frac{10}{c - 2}
\end{align}

A afirmação é válida para $c = 3$ e  $n_0 = 10$.

\section{Paradigma divisão e conquista}

Muitos algoritmos importantes são estruturalmente recursivos. Tais algoritmos seguem tipicamente o paradigma da divisão e conquista (CLR):
\begin{itemize}
    \item Quebram o problema original em pedaços menores.
    \item Resolvem os subproblemas recursivamente.
    \item Combinam estas soluções para criar uma solução para o problema original.
\end{itemize}

Então, existem 3 passos essenciais para o paradigma: 
\begin{enumerate}
    \item Dividir
    \item Conquistar
    \item Combinar
\end{enumerate}

\section{Resolvendo a Torre de Hanoi}
Supondo três torres, E, C, D, com a torre E possui N discos.
O objetivo é mover N discos de E para C.

\begin{itemize}
    \item Nunca colocar 1 disco maior sobre um disco menor.
    \item Nunca mover mais que 1 disco por vez.
\end{itemize}
\end{document}
