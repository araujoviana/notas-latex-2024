% Created 2024-10-31 qui 10:43
% Intended LaTeX compiler: pdflatex
\documentclass[11pt]{article}
\usepackage[utf8]{inputenc}
\usepackage[T1]{fontenc}
\usepackage{graphicx}
\usepackage{longtable}
\usepackage{wrapfig}
\usepackage{rotating}
\usepackage[normalem]{ulem}
\usepackage{amsmath}
\usepackage{amssymb}
\usepackage{capt-of}
\usepackage{hyperref}
\author{Matheus Gabriel}
\date{\textit{<2024-10-31 qui>}}
\title{Estrutura de Dados I\\\medskip
\large Lista duplamente encadeada}
\hypersetup{
 pdfauthor={Matheus Gabriel},
 pdftitle={Estrutura de Dados I},
 pdfkeywords={},
 pdfsubject={},
 pdfcreator={Emacs 29.4 (Org mode 9.7.11)}, 
 pdflang={English}}
\begin{document}

\maketitle
\tableofcontents

\section{Introdução à lista duplamente encadeada}
\label{sec:orgb87dddc}

Na lista duplamente encadeada (diferente da simplesmente encadeada) o nó possui um endereço tanto pra trás dele quanto para frente, e além do \emph{head} que aponta para o primeiro elemento, também temos obrigatoriamente o \emph{tail} que sempre aponta para o último elemento.
\subsection{Estrutura Visual com Graphviz}
\label{sec:org7d7e243}

\begin{center}
\includegraphics[width=.9\linewidth]{linked_list_double.png}
\label{}
\end{center}
\subsection{Estrutura Visual Circular com Graphviz}
\label{sec:org36f1bfb}

\begin{center}
\includegraphics[width=.9\linewidth]{circular_linked_list.png}
\label{}
\end{center}
\subsection{Estrutura Visual da Lista Duplamente Encadeada Circular com Graphviz}
\label{sec:orgd5accfd}

\begin{center}
\includegraphics[width=.9\linewidth]{circular_doubly_linked_list.png}
\label{}
\end{center}

A imagem ta meio errada mas a ideia permanece
\subsection{Estrutura}
\label{sec:org0e57cab}

No caso, a estrutura seria:
\begin{enumerate}
\item \texttt{previous}
\item \texttt{conteúdo}
\item \texttt{next}
\end{enumerate}
Além disso, opcionalmente é possivel ter um cursor \textbf{current} que armazena o nó atual sendo verificado.
\subsubsection{Importante, estilo pergunta de prova}
\label{sec:org9b06136}
\begin{quote}
Os TADs da lista circular e da lista [circular] duplamente encadeada
são similares ao TAD da lista simplesmente encadeada.
\end{quote}

Se eu alterar uma lista simplesmente encadeada por uma lista circular duplamente encadeada (por exemplo), o funcionamento não deveria quebrar, embora a abordagem interna mude um pouco (por motivos óbvios né, são tipos de lista diferentes).

Outra coisa, importante saber o que o \texttt{static} faz
\end{document}
