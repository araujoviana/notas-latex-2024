\documentclass{article}
\usepackage{graphicx} % Required for inserting images
\usepackage{minted}
\usepackage{hyperref}
\usepackage{tcolorbox}
\usepackage{listings}


\lstset{
    basicstyle=\ttfamily,
    keywordstyle=\color{blue},
    commentstyle=\color{green},
    stringstyle=\color{red},
    frame=single,
}


\title{Estrutura de Dados I
\\ \large Lista (Encadeada ou em inglês, Linked List)}
\author{Matheus Gabriel}
\date{Outubro 2024}

\begin{document}

\maketitle

\section{Definição}

Costuma ser chamada apenas de lista, ela é:
\begin{enumerate}
    \item Conjunto de dados do mesmo tipo
    \item \textbf{Não} Usamos indices para acessar os seus elementos, usamos \textbf{nós}
\end{enumerate}

\section{Nós (nodes)}

Para acessar elementos na lista encadeada usamos \textbf{nós}, eles são alocados dinamicamente, conforme a necessidade. E também eles não precisam estar localizados sequencialmente na memória.

\section{Vantagens e Desvantagens}

Vantagens:
\begin{enumerate}
    \item Tamanho dinâmico
    \item Não há deslocamento de memória na inserção de elementos
    \item Remoção de elementos NÃO deixam "buracos" nos arrays
\end{enumerate}

Desvantagens:
\begin{enumerate}
    \item Nós não são sequenciais na memória
    \item Custo extra de memória (\textit{Overhead})
\end{enumerate}

\section{Composição}
Um nó da lista contém duas partes:
\item 
\begin{enumerate}
    \item data $\rightarrow$ os dados armazenados no nó
    \item *next $\rightarrow$ um ponteiro/referência para o próximo nó da lista (já que nós são armazenados sequencialmente na memória).
\end{enumerate}

Visualmente eles podem serem representados como:
\begin{figure}[h!]
    \centering
    \includegraphics[width=0.5\linewidth]{image.png}
\end{figure}

\section{Implementação}

Não é necessária a criação de uma classe Linked List, pode ter apenas o Nó \textit{head}.

\subsection{Apenas o nó}
\subsubsection{C}

\begin{lstlisting}[language=C]
struct Node {
    char data;
    struct Node* next;
};
\end{lstlisting}

Repare que next é um ponteiro para outro Node (pode ser qualquer Node, incluindo ele mesmo). O tipo char no data é só um exemplo.

\subsubsection{Java}
\begin{lstlisting}[language=Java]
public class Node {
    private char data;
    private Node next;

    // Constructor, getter, setter, etc...
}
\end{lstlisting}

Mais simples, porém a implementação é semelhante.

\subsection{Usando a classe Linked List}

\begin{lstlisting}[language=Java]
public class LinkedList {
    // Primeiro no
    private Node head;

    // OPCIONAL
    // Ultimo no
    private Node tail;
    // Contador de nos
    private int count;
}
\end{lstlisting}

Na verdade, como nos códigos anteriores, a única parte necessária é o primeiro nó.
\end{document}
