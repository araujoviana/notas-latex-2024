\documentclass{article}
\usepackage{graphicx} % Required for inserting images
\usepackage{minted}
\usepackage{hyperref}
\usepackage{tcolorbox}


\title{Estrutura de Dados I
\\ \large Pilhas}
\author{Matheus Gabriel}
\date{Agosto 2024}

\begin{document}

\maketitle

\section{Pilhas}
\subsection{Introdução}
A pilha \textbf{(stack)} é uma estrutura de dados que define o acesso de dados usando o principio LIFO:

\newtcolorbox[auto counter, number within=section]{definicao}[2][]{%
  colback=blue!5!white, colframe=blue!75!black, 
  fonttitle=\bfseries, title=Definição~\thetcbcounter: #2, 
  sharp corners=south, boxrule=0.8mm, arc=4mm, top=2mm, bottom=2mm, #1
}

\begin{definicao}[label={def:LIFO}]{LIFO}
  \textbf{LIFO (Último a Entrar, Primeiro a Sair):} Um método de gerenciamento de dados onde o último elemento adicionado é o primeiro a ser removido. Comumente utilizado em estruturas de dados do tipo pilha, onde as operações são realizadas na ordem último a entrar, primeiro a sair.

  \begin{itemize}
    \item \textbf{Operação Push:} Adiciona o elemento \( x \) ao topo da pilha.
    \item \textbf{Operação Pop:} Remove o elemento do topo da pilha, que é \( x \).
  \end{itemize}
\end{definicao}

\subsection{Métodos comuns para pilhas}

\begin{table}[h!]
\centering
\begin{tabular}{|l|l|l|}
\hline
\textbf{Método} & \textbf{Uso} & \textbf{Exemplo} \\
\hline
\texttt{create()} & Cria uma nova pilha & \texttt{Stack<Integer> stack = new Stack<>();} \\
\texttt{push(E e)} & Adiciona elemento ao topo & \texttt{stack.push(10);} \\
\texttt{pop()} & Remove e retorna o topo & \texttt{int item = stack.pop();} \\
\texttt{peek()} & Retorna o topo sem remover & \texttt{int top = stack.peek();} \\
\texttt{isEmpty()} & Verifica se está vazio & \texttt{boolean empty = stack.isEmpty();} \\
\texttt{isFull()} & Verifica se está cheio & \texttt{boolean full = stack.isFull();} \\
\texttt{size()} & Retorna o número de elementos & \texttt{int size = stack.size();} \\
\texttt{count()} & Contagem de elementos (alternativa) & \texttt{int count = stack.count();} \\
\texttt{clear()} & Remove todos os elementos & \texttt{stack.clear();} \\
\hline
\end{tabular}
\label{tab:stack_methods}
\end{table}

\subsection{Implementação do zero}
Essa implementação provavelmente está incompleta.

\begin{minted}{java}
// Stack.java
public class Stack {
    private int[] data;
    private int count;

    // Construtor para inicializar a pilha com um tamanho específico
    public Stack(int size) {
        data = new int[size];
        count = 0;
    }

    // Adiciona um elemento ao topo da pilha
    public void push(int value) {
        if (count == data.length) {
            System.out.println("Erro: pilha cheia.");
            return;
        }
        data[count++] = value;
    }

    // Remove e retorna o elemento do topo da pilha
    public int pop() {
        if (isEmpty()) {
            System.out.println("Erro: pilha vazia.");
            throw new RuntimeException("Pilha vazia");
        }
        return data[--count];
    }

    // Retorna o elemento do topo da pilha sem removê-lo
    public int peek() {
        if (isEmpty()) {
            System.out.println("Erro: pilha vazia.");
            throw new RuntimeException("Pilha vazia");
        }
        return data[count - 1];
    }

    // Verifica se a pilha está vazia
    public boolean isEmpty() {
        return count == 0;
    }

    // Retorna o número de elementos na pilha
    public int size() {
        return count;
    }

    // Remove todos os elementos da pilha
    public void clear() {
        count = 0;
    }

    // Representa a pilha como uma string
    @Override
    public String toString() {
        StringBuilder sb = new StringBuilder("Pilha: [");
        for (int i = 0; i < count; i++) {
            sb.append(data[i]);
            if (i < count - 1) {
                sb.append(", ");
            }
        }
        sb.append("]");
        return sb.toString();
    }

    // Método principal para teste
    public static void main(String[] args) {
        Stack stack = new Stack(5);
        
        // Adiciona elementos à pilha
        stack.push(1);
        stack.push(2);
        stack.push(3);

        // Exibe a pilha
        System.out.println(stack);

        // Exibe o topo da pilha
        System.out.println("Topo da pilha: " + stack.peek());

        // Remove e exibe elementos da pilha
        while (!stack.isEmpty()) {
            System.out.println("Removido: " + stack.pop());
        }

        // Exibe se a pilha está vazia
        System.out.println("A pilha está vazia? " + stack.isEmpty());

        // Exibe a pilha após limpar
        stack.clear();
        System.out.println(stack);
    }
}

\end{minted}
\end{document}
