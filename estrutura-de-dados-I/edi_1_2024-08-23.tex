\documentclass{article}
\usepackage{graphicx} % Required for inserting images
\usepackage{minted}
\usepackage{hyperref}
\usepackage{tcolorbox}


\title{Estrutura de Dados I
\\ \large Continuação de Tipos de Dados em Java}
\author{Matheus Gabriel}
\date{Agosto 2024}

\begin{document}

\maketitle

\section{Tipo abstrato de dado}

\subsection{Definição}

\begin{quote}
    "É algo que existe como ideia ou conceito, mas talvez não concretizado."
\end{quote} 

Eles são definidos como a aplicação da abstração sobre um conceito real ou ideia.

\begin{tcolorbox}[title=Importante, colback=yellow!10!white, colframe=orange!75!black]
Ele é baseado em um tipo de dado e:

\begin{itemize}
    \item Especifica \textbf{o que} cada operação realiza.
    \item \textit{Não} especifica como a operação realiza sua tarefa.
\end{itemize}

\end{tcolorbox}

\subsection{Benefícios dos TADs}

\begin{itemize}
    \item Encapsulamento
    \subitem O código das operações fica (quase) oculto ao usuário.
    \item Abstração
    \subitem O usuário não precisa entender a implementação.
    \item Reuso
    \subitem Os TADs podem ser usados em  múltiplos projetos.
    \item Segurança
    \subitem Restringe o acesso e a manipulação direta dos dados.
    \item Manutenção
    \subitem A organização e abstração facilita a manutenção futura.
    \item Padronização
    \subitem Um TAD bem projetado segue um padrão, o que facilita a escrita de código legível e consistente.
\end{itemize}

\end{document}
