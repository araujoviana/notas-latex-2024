\documentclass{article}
\usepackage{graphicx} % Required for inserting images
\usepackage{minted}
\usepackage{hyperref}
\usepackage{tcolorbox}


\title{Estrutura de Dados I
\\ \large Tipos de Dados em Java}
\author{Matheus Gabriel}
\date{Agosto 2024}

\begin{document}

\maketitle

\section{Tipos de variáveis}
\subsection{Tipos primitivos}
São os tipos criados pela própria linguagem de programação. \\
Cada tipo ocupa uma quantidade de bytes diferente:


\begin{table}[h!]
    \centering
    \begin{tabular}{|c|c|l|}
        \hline
        \textbf{Tipo de Dados} & \textbf{Tamanho (Bytes)} & \textbf{Descrição simplificada} \\
        \hline
        byte & 1 & Número inteiro pequeno \\
        short & 2 & Número inteiro médio \\
        int & 4 & Número inteiro \\
        long & 8 & Número inteiro grande \\
        float & 4 & Número real de precisão simples \\
        double & 8 & Número real de precisão dupla \\
        char & 2 & Carácter Unicode \\
        boolean & 1 (ou 1 bit, em certos casos) & Valor lógico (true/false) \\
        \hline
    \end{tabular}
\end{table}


Ou use a \href{https://www.w3schools.com/java/java_data_types.asp}{tabela do W3Schools} como referência.
\subsection{Tipos estruturados (classes)}
São criados a partir da composição de tipos primitivos. O maior exemplo são as próprias classes.

\begin{minted}[
linenos
]
{java}
// Estudante.java
public class Estudante {
    private String matricula;
    private String nome;
    private float n1;
    private float n2;
    private float sub;
    private float pf;
}
\end{minted}
Ao invés de armazenar as variáveis separadamente eles ficam juntas dentro da classe \texttt{Estudante}, o que ajuda na organização e manutenção do código.

\begin{tcolorbox}[colback=yellow!10!white, colframe=yellow!75!black, title=Aviso]
O conceito de visibilidade utilizado por \texttt{public} e \texttt{private} será explicado em anotações futuras, por enquanto eles podem ser ignorados.
\end{tcolorbox}

\clearpage

\section{Declaração de classes}
O código a seguir declara os objetos \texttt{estudante} e \texttt{st}, é importante que o programa saiba aonde essa classe está armazenada, então por conveniência todos os arquivos \texttt{.java} desse exemplo estão no mesmo diretório.
\begin{minted}[linenos]{java}
// Main.java
public class Main {
    public static void main(String[] args) {
        Estudante estudante = new Estudante();
        Estudante st;
        st = new Estudante();
    }
}    
\end{minted}

\clearpage

\section{Getters e Setters}
Os métodos \textit{getters} e \textit{setters} são usados para acessar um valor e definir um valor respectivamente. A norma para a nomenclatura desses tipos de métodos segue o estilo do exemplo a seguir:
\begin{minted}[linenos]{java}
// Estudante.java
public class Estudante {
    private String matricula;
    private String nome;
    private float n1;
    private float n2;
    private float sub;
    private float pf;

    public String getMatricula() {
        return matricula;
    }

    public void setMatricula(String matricula) {
        this.matricula = matricula;
    }

}
\end{minted}

\begin{minted}[linenos]{java}
// Main.java
public class Main {
    public static void main(String[] args) {
        Estudante estudante = new Estudante();
        Estudante st;
        st = new Estudante();

        estudante.setMatricula("PRIMEIRO");

        System.out.println(estudante.getMatricula());
    }
}

\end{minted}

\begin{tcolorbox}[colback=yellow!10!white, colframe=yellow!75!black, title=Aviso]
Métodos são "praticamente" semelhantes às funções, mas elas estão sempre presentes dentro de uma classe.
\end{tcolorbox}

\subsection{Uso do \texttt{this.}}
O \texttt{this.} é usado para alterar a variável da instância criada (\texttt{estudante} no caso do exemplo anterior), se usarmos \texttt{matricula} diretamente alteramos o valor do próprio parâmetro, o que pode causar erros em operações futuras com essa mesma classe.

\subsection{Exemplo de setter modificado}
O codigo interno dos getters e setters não precisam seguir o mesmo modelo, no exemplo a seguir, alteramos o setter para que o valor (da instância) nunca seja negativo.
\begin{minted}[linenos]{java}
// InteiroPositivo.java
public class InteiroPositivo {
    private int valor;

    public int getValor() {
        return valor;
    }

    public void setValor(int valor) {
        if (valor < 0) {
            this.valor = -valor;
        }
        else {
            this.valor = valor;
        }
    }
}
\end{minted}
\end{document}
