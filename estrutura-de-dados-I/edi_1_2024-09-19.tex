\documentclass{article}
\usepackage{graphicx} % Required for inserting images
\usepackage{minted}
\usepackage{hyperref}
\usepackage{tcolorbox}


\title{Estrutura de Dados I
\\ \large Deque (fila dupla)}
\author{Matheus Gabriel}
\date{Agosto 2024}

\begin{document}

\maketitle

\section{Definição}

Uma \textbf{fila dupla} é uma fila que permite a inserção e remoçãos dos elementos nas duas extremidades da fila.

Importante:
\begin{enumerate}
    \item Ela não impõe a regra FIFO
    \item Não é LIFO e FIFO
    \item Há uma restrição: trabalhamos somente com as extremidades
\end{enumerate}

\section{Comandos}

\begin{enumerate}
    \item 
    \begin{verbatim}
        InsertFront(deque,elem)
    \end{verbatim}
     Insere o elemento elem no início da fila dupla deque 
    \item 
    \begin{verbatim}
        InsertBack(deque,elem) 
    \end{verbatim}
    Insere o elemento elem no fim da fila dupla deque
    \item 
    \begin{verbatim}
        RemoveFront(deque) 
    \end{verbatim}
    Remove o elemento elem no início da fila dupla deque
    \item 
    \begin{verbatim}
        RemoveBack(deque) 
    \end{verbatim}
    Remove o elemento elem no final da fila dupla deque
        \item 
    \begin{verbatim}
        IsFull(deque) 
    \end{verbatim}
    Verifica se o deque está cheio
        \item 
    \begin{verbatim}
        IsEmpty(deque) 
    \end{verbatim}
    Verifica se o deque está vazio
        \item 
    \begin{verbatim}
        Clear(deque) 
    \end{verbatim}
    Limpa o deque
\end{enumerate}
\end{document}
