\documentclass{article}
\usepackage{graphicx} % Required for inserting images
\usepackage{minted}
\usepackage{hyperref}
\usepackage{tcolorbox}


\title{Estrutura de Dados I
\\ \large Pilhas}
\author{Matheus Gabriel}
\date{Agosto 2024}

\begin{document}

\maketitle

\section{Continuando pilhas}

\subsection{Mais sobre a implementação do zero}

\subsubsection{Relembrando o código}

\begin{minted}{java}
// Stack.java
public class Stack {
    private int[] data;
    private int count;

    // Construtor para inicializar a pilha com um tamanho específico
    public Stack(int size) {
        data = new int[size];
        count = 0;
    }

    // Adiciona um elemento ao topo da pilha
    public void push(int value) {
        if (count == data.length) {
            System.out.println("Erro: pilha cheia.");
            return;
        }
        data[count++] = value;
    }

    // Remove e retorna o elemento do topo da pilha
    public int pop() {
        if (isEmpty()) {
            System.out.println("Erro: pilha vazia.");
            throw new RuntimeException("Pilha vazia");
        }
        return data[--count];
    }

    // Retorna o elemento do topo da pilha sem removê-lo
    public int peek() {
        if (isEmpty()) {
            System.out.println("Erro: pilha vazia.");
            throw new RuntimeException("Pilha vazia");
        }
        return data[count - 1];
    }

    // Verifica se a pilha está vazia
    public boolean isEmpty() {
        return count == 0;
    }

    // Retorna o número de elementos na pilha
    public int size() {
        return count;
    }

    // Remove todos os elementos da pilha
    public void clear() {
        count = 0;
    }

    // Representa a pilha como uma string
    @Override
    public String toString() {
        StringBuilder sb = new StringBuilder("Pilha: [");
        for (int i = 0; i < count; i++) {
            sb.append(data[i]);
            if (i < count - 1) {
                sb.append(", ");
            }
        }
        sb.append("]");
        return sb.toString();
    }

    // Método principal para teste
    public static void main(String[] args) {
        Stack stack = new Stack(5);
        
        // Adiciona elementos à pilha
        stack.push(1);
        stack.push(2);
        stack.push(3);

        // Exibe a pilha
        System.out.println(stack);

        // Exibe o topo da pilha
        System.out.println("Topo da pilha: " + stack.peek());

        // Remove e exibe elementos da pilha
        while (!stack.isEmpty()) {
            System.out.println("Removido: " + stack.pop());
        }

        // Exibe se a pilha está vazia
        System.out.println("A pilha está vazia? " + stack.isEmpty());

        // Exibe a pilha após limpar
        stack.clear();
        System.out.println(stack);
    }
}


\end{minted}

\subsubsection{StringBuilder}

\begin{minted}{java}
    // Representa a pilha como uma string
    @Override
    public String toString() {
        StringBuilder sb = new StringBuilder("Pilha: [");
        for (int i = 0; i < count; i++) {
            sb.append(data[i]);
            if (i < count - 1) {
                sb.append(", ");
            }
        }
        sb.append("]");
        return sb.toString();
    }
\end{minted}
Para manter a eficiência sempre use \mintinline{java}{StringBuilder} ao invés de usar uma string temporária no seu método \mintinline{java}{toString()}.
\end{document}
